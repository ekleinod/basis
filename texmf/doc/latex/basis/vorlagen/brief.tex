\documentclass[ngerman]{scrlttr2}

\usepackage{basbrief}
%\usepackage[draft|final, font=charter|droid|hfold|mathpazo|original|times, fontsize=<groesse>, hypercolor=<color>, hyperdriver=<driver>, layout=kopfzeile|infospalte|infospaltefett, nobackaddress, nofoldmarks, oneside, pagestyle=beides|fuss|fussseite|leer]{basbrief}

\author{Autorname}
%\briefkopf{Briefkopf}

%\strasse{Straße}
%\plz{PLZ}
%\ort{Ort}
%\telefon{(030) 1\,22\,33\,44}
%\email{abc@abc.de}
%\adresszusatz{Adresszusatz}

%\logo{logo}

\begin{document}

	%\renewcommand{\enclname}{Anlage} % Standard: Anlagen, wenn nur eine Anlage gemeint ist: diese Zeile entkommentieren

	% hier den Empfänger (\\-getrennt) eintragen
	\begin{letter}{Empfänger\\Empfängerstraße\\Empfängerort}

		%\setkomavar{subject}{Betreff} % Betreff
		%\setkomavar{date}{27.\ November 2013} % Datum (default: aktuelles Datum)
		%\setkomavar{signature}{} % Unterschrift (default: Name)

		%\setkomavar{yourref}{} % Geschäftszeile: Ihr Zeichen
		%\setkomavar{yourmail}{} % Geschäftszeile: Ihr Schreiben vom
		%\setkomavar{myref}{} % Geschäftszeile: Unser Zeichen
		%\setkomavar{invoice}{} % Geschäftszeile: Rechnungsnummer

		%\newkomavar*[Meins]{myadd} % Geschäftszeile: neues Feld myadd mit Titel "Meins"
		%\setkomavar{myadd}{Inhalt} % Geschäftszeile: neues Feld

		\opening{Hallo \dots,}

			hier steht der Brieftext.

		\closing{Mit freundlichen Grüßen,}

		%\ps % Post-Scriptum
		%\encl{} % Anlage(n)
		%\cc{} % Verteiler

	\end{letter}

\end{document}

