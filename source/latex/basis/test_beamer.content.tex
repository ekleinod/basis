\usepackage[pangram]{blindtext}

\subject{Thema}
\title{Titel}
\subtitle{Untertitel}
\author{Autor 1\and Autor 2\and Autor 3}
\date{6. Januar 2016}

\AtBeginLecture{\frame{\Large Heute: \insertlecture}}

\begin{document}

	%\includeonlylecture{abend2}



	\mode<presentation>{
		\frame{\titlepage}
	}
	\mode<article>{
		\maketitle
	}

	\begin{frame}
		Nicht vergessen: mit \hologo{pdfLaTeX} oder \hologo{XeLaTeX} übersetzen.
	\end{frame}

	\begin{frame}
		\begin{abstract}
			Diese Datei zeigt die verschiedenen Möglichkeiten mit Beamer.
			Diese Datei zeigt die verschiedenen Möglichkeiten mit Beamer.
			Diese Datei zeigt die verschiedenen Möglichkeiten mit Beamer.
			Diese Datei zeigt die verschiedenen Möglichkeiten mit Beamer.
			Diese Datei zeigt die verschiedenen Möglichkeiten mit Beamer.
		\end{abstract}
	\end{frame}

	\section*{Agendas}

	\begin{frame}{Agenda}
		\tableofcontents

		Leer, da Parts benutzt wurden.
	\end{frame}

	\subsection*{Teil 1: der erste Abend}

	\begin{frame}{Agenda Teil 1}
		\tableofcontents[part=1, subsectionstyle=hide]
	\end{frame}

	\subsection*{Teil 2: der zweite Abend}

	\begin{frame}{Agenda Teil 2 zum Aufklappen}
		\tableofcontents[part=2, pausesections]
	\end{frame}



	\lecture{Erster Abend: SRaT}{abend1}


	\part{Teil 1: der erste Abend}

	\frame{\partpage}

	\section{Normaler Text}

	\begin{frame}{Text}
		Franz jagt im komplett verwahrlosten Taxi quer durch Bayern.
		Zwölf Boxkämpfer jagen Viktor quer über den großen Sylter Deich.
		Vogel Quax zwickt Johnys Pferd Bim.
		Sylvia wagt quick den Jux bei Pforzheim.
	\end{frame}

	\section{Besondere Umgebungen}

	\begin{frame}{Verse}
		\begin{verse}
			Franz jagt im komplett verwahrlosten Taxi quer durch Bayern.
			Zwölf Boxkämpfer jagen Viktor quer über den großen Sylter Deich.

			Vogel Quax zwickt Johnys Pferd Bim.
			Sylvia wagt quick den Jux bei Pforzheim.

			Prall vom Whisky flog Quax den Jet zu Bruch.\\
			Jeder wackere Bayer vertilgt bequem zwo Pfund Kalbshaxen.\\
			Stanleys Expeditionszug quer durch Afrika wird von jedermann bewundert.\\
		\end{verse}
	\end{frame}

	\begin{frame}{Quotation}
		\begin{quotation}
			Franz jagt im komplett verwahrlosten Taxi quer durch Bayern.
			Zwölf Boxkämpfer jagen Viktor quer über den großen Sylter Deich.

			Vogel Quax zwickt Johnys Pferd Bim.
			Sylvia wagt quick den Jux bei Pforzheim.

			Prall vom Whisky flog Quax den Jet zu Bruch.
			Jeder wackere Bayer vertilgt bequem zwo Pfund Kalbshaxen.
			Stanleys Expeditionszug quer durch Afrika wird von jedermann bewundert.
		\end{quotation}
	\end{frame}

	\begin{frame}{Quote}
		\begin{quote}
			Franz jagt im komplett verwahrlosten Taxi quer durch Bayern.
			Zwölf Boxkämpfer jagen Viktor quer über den großen Sylter Deich.

			Vogel Quax zwickt Johnys Pferd Bim.
			Sylvia wagt quick den Jux bei Pforzheim.

			Prall vom Whisky flog Quax den Jet zu Bruch.
			Jeder wackere Bayer vertilgt bequem zwo Pfund Kalbshaxen.
			Stanleys Expeditionszug quer durch Afrika wird von jedermann bewundert.
		\end{quote}
	\end{frame}

	\begin{frame}<1>[plain, label=end]{Ende}
		\Huge{Ende.}
	\end{frame}






	\lecture{Zweiter Abend: OSR}{abend2}


	\part{Teil 2: der zweite Abend}

	\frame{\partpage}

	\section{Normaler Text}

	\begin{frame}{Text}
		Franz jagt im komplett verwahrlosten Taxi quer durch Bayern.
		Zwölf Boxkämpfer jagen Viktor quer über den großen Sylter Deich.
		Vogel Quax zwickt Johnys Pferd Bim.
		Sylvia wagt quick den Jux bei Pforzheim.
	\end{frame}

	\begin{frame}{Langer Text}
		Prall vom Whisky flog Quax den Jet zu Bruch.
		Jeder wackere Bayer vertilgt bequem zwo Pfund Kalbshaxen.
		Stanleys Expeditionszug quer durch Afrika wird von jedermann bewundert.
		Franz jagt im komplett verwahrlosten Taxi quer durch Bayern.

		Zwölf Boxkämpfer jagen Viktor quer über den großen Sylter Deich.
		Vogel Quax zwickt Johnys Pferd Bim.
		Sylvia wagt quick den Jux bei Pforzheim.
		Prall vom Whisky flog Quax den Jet zu Bruch.
	\end{frame}

	\begin{frame}{Text mit Auszeichnungen und Fußnoten}
		Prall vom \emph{Whisky}\footnote{emph} flog \textbf{Quax}\footnote{textbf} den Jet zu Bruch.
		Jeder \textit{wackere}\footnote{textit} Bayer \textsl{vertilgt}\footnote{textsl} bequem zwo Pfund Kalbshaxen.
		Stanleys \alert{Expeditionszug}\footnote{alert} quer durch Afrika wird von jedermann bewundert.
	\end{frame}

	\begin{frame}{Ein Zitat (quote)}
		Prall vom Whisky flog Quax den Jet zu Bruch.

		\begin{quote}
			Jeder wackere Bayer vertilgt bequem zwo Pfund Kalbshaxen.

			Stanleys Expeditionszug quer durch Afrika wird von jedermann bewundert.
			Franz jagt im komplett verwahrlosten Taxi quer durch Bayern.
		\end{quote}

		Zwölf Boxkämpfer jagen Viktor quer über den großen Sylter Deich.
	\end{frame}

	\begin{frame}{Ein Zitat (quotation)}
		Prall vom Whisky flog Quax den Jet zu Bruch.

		\begin{quotation}
			Jeder wackere Bayer vertilgt bequem zwo Pfund Kalbshaxen.

			Stanleys Expeditionszug quer durch Afrika wird von jedermann bewundert.
			Franz jagt im komplett verwahrlosten Taxi quer durch Bayern.
		\end{quotation}

		Zwölf Boxkämpfer jagen Viktor quer über den großen Sylter Deich.
	\end{frame}

	\section{Aufzählungen}

	\begin{frame}{Agenda für diese Section}
		\tableofcontents[currentsection]
	\end{frame}

	\subsection{Aufzählungen ohne Nummern}

	\begin{frame}{Unnumeriert}
		\begin{itemize}
			\item Erster Listenpunkt, Stufe 1
			\item Zweiter Listenpunkt, Stufe 1
			\item Dritter Listenpunkt, Stufe 1
		\end{itemize}
	\end{frame}

	\begin{frame}{Unnumeriert mit Pause}
		\begin{itemize}
			\item Erster Listenpunkt, Stufe 1
				\pause
			\item Zweiter Listenpunkt, Stufe 1
				\pause
			\item Dritter Listenpunkt, Stufe 1
		\end{itemize}
	\end{frame}

	\begin{frame}{Unnumeriert mit hervorgehobenem Aufdecken}
		\begin{itemize}
			\item<+-| alert@+> Erster Listenpunkt
			\item<+-| alert@+> Zweiter Listenpunkt
			\item<+-| alert@+> Dritter Listenpunkt
			\item<+-| alert@+> Vierter Listenpunkt
		\end{itemize}
	\end{frame}

	\begin{frame}{Unnumeriert mit hervorgehobenem Aufdecken (vereinfacht)}
		\begin{itemize}[<+-| alert@+>]
			\item Erster Listenpunkt
			\item Zweiter Listenpunkt
			\item Dritter Listenpunkt
			\item Vierter Listenpunkt
		\end{itemize}
	\end{frame}

	\subsection{Aufzählungen mit Nummern}

	\begin{frame}{Numeriert}
		\begin{enumerate}
			\item Erster Listenpunkt, Stufe 1
			\item Zweiter Listenpunkt, Stufe 1
			\item Dritter Listenpunkt, Stufe 1
		\end{enumerate}
	\end{frame}

	\subsection{Aufzählungen mit Beschreibungstexten}

	\begin{frame}{Beschreibung}
		\begin{description}
			\item[Erster] Listenpunkt, Stufe 1
			\item[Zweiter] Listenpunkt, Stufe 1
			\item[Dritter] Listenpunkt, Stufe 1
		\end{description}
	\end{frame}

	\section{Grafiken}

	\begin{frame}{Eingebundenes PDF}
		\includegraphics{testlogo}
	\end{frame}

	\begin{frame}{Eingebundenes PDF skaliert}
		\includegraphics[width=\textwidth]{testlogo}
	\end{frame}

	% Modus benutzen, da sonst Übersetzungsfehler bei Artikeln
	\mode<presentation>{
		\AtBeginNote{Anfangsanmerkung für alle Bilder.}
		\AtEndNote{Endanmerkung für alle Bilder.}
	}
	\begin{frame}{Eingebundenes PDF aufpoppend mit gemeinsamen Anmerkungen}
		\includegraphics<1->{testlogo}
		\note[item]{Eine Anmerkung zum ersten Bild.}
		\note[item]{Eine zweite Anmerkung zum ersten Bild.}
		\includegraphics<2->{testlogo}
		\note[item]{Eine Anmerkung zum zweiten Bild.}
		\includegraphics<3->{testlogo}
	\end{frame}

	\begin{frame}{Eingebundenes PDF aufpoppend mit getrennten Anmerkungen}
		\includegraphics<1->{testlogo}
		\note[item]<1>{Eine Anmerkung zum ersten Bild.}
		\note[item]<1>{Eine zweite Anmerkung zum ersten Bild.}
		\includegraphics<2->{testlogo}
		\note[item]<2>{Eine Anmerkung zum zweiten Bild.}
		\includegraphics<3->{testlogo}
	\end{frame}
	% wieder ausschalten
	\mode<presentation>{
		\AtBeginNote{}
		\AtEndNote{}
	}

	\begin{frame}{Eingebundenes PDF als Figure (1)}
		\begin{figure}
			\includegraphics{testlogo}
			\caption{Bildunterschrift unter Bild.}
		\end{figure}
		\note[item]{Anmerkung dazu.}
	\end{frame}

	\begin{frame}{Eingebundenes PDF als Figure (2)}
		\begin{figure}
			\caption{Bildunterschrift über Bild.}
			\includegraphics{testlogo}
		\end{figure}
	\end{frame}

	\section{Spezielles}

	\begin{frame}{Definition und Beispiel}
		\begin{definition}
			Hier ist eine Definition.
		\end{definition}
		\begin{example}
			Hier ist ein Beispiel dazu.
		\end{example}
	\end{frame}

	\begin{frame}{Satz und Beweis mit Aufblättern}
		\begin{theorem}
			Eine Behauptung.
		\end{theorem}
		\begin{proof}
			\begin{enumerate}
				\item<1-> Sei es so.
				\item<2-> Dann folgt dies daraus.
				\item<1-> Damit gilt das immer.\qedhere
			\end{enumerate}
		\end{proof}
		\uncover<3->{Ui.}
	\end{frame}

	\begin{frame}{Horizontale Blöcke}
		\begin{block}{Block 1}
			Inhalt des ersten Blocks.
		\end{block}
		\begin{block}{Block 2}
			Inhalt des zweiten Blocks.

			Auch Inhalt des zweiten Blocks.
		\end{block}
	\end{frame}

	\begin{frame}{Vertikale Blöcke (Spalten)}
		\begin{columns}
			\column{.3\textwidth}
				\begin{block}{Block 1}
					Inhalt des ersten Blocks.
				\end{block}
			\column{.7\textwidth}
				\begin{block}{Block 2}
					Inhalt des zweiten Blocks.

					Auch Inhalt des zweiten Blocks.
				\end{block}
		\end{columns}
	\end{frame}

\begin{frame}[fragile]{Code und anderer Verbatimtext.}
\begin{verbatim}
function returnNull() {
return null;
}
\end{verbatim}
\begin{uncoverenv}<2>
Note the use of \verb|null|.
\end{uncoverenv}
\end{frame}
% wichtig: nicht einrücken, sonst Fehler

	\begin{frame}[label=mei1]{Label und Referenz auf sich selbst}
		Hallo, ich bin Frame~\autoref{mei1}.
	\end{frame}

	\mode<article>{
		\fbox{\pgfimage[page=\insertframenumber, width=.5\textwidth]{test_beamer.slides}}
	}

	\begin{frame}<1>[label=wdh]{Wiederholter Frame.}

		Hier ist Text, der gleich wiederholt wird.

		\begin{overprint}
			\onslide<1|handout:0>
			Hier ist Text, der nur beim ersten Mal angezeigt wird.
			Und ein Link zur \hyperlink{wdh<2>}{\beamergotobutton{Wiederholung}}

			\onslide<2-|handout:0>
			Dieser Text wird erst bei der Wiederholung angezeigt!
		\end{overprint}
	\end{frame}

	Hier sind wir kurz hinter Frame \insertframenumber.

	\begin{frame}{Zwischentext.}
		Text dazwischen. Gleich\dots
	\end{frame}

	Hier sind wir kurz hinter Frame \insertframenumber.

	\againframe<2>{wdh}

	Hier sind wir kurz hinter Frame \insertframenumber.

	\againframe<2>{end}

	\mode<article>{
		\fbox{\includeslide[width=.3\textwidth]{end}}
	}

	Hier sind wir kurz hinter Frame \insertframenumber.






	\appendix

	\section{\appendixname}

	\frame{\tableofcontents}

	\subsection{Zusatzmaterial}

	\frame{Details}
	\frame{Text, der im Hauptteil fehlt.}

	\subsection{Noch mehr Zusatzmaterial}

	\frame{Mehr Details.}


\end{document}

