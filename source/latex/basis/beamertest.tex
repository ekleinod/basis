\mode<presentation> {
	%\usetheme{Singapore}
	%\usetheme{Berkeley}
	%\usetheme{Rochester}
	\usetheme{basis}

	%\usecolortheme{dove}
	\usecolortheme{seahorse}

	%\usepackage{beamerthemesplit}

	\setbeamercovered{transparent}
}

\usepackage[ngerman]{babel}
\usepackage{tikz}

\title[Heiterkeit]{Eine wunderbare Heiterkeit}

\subtitle{Gleich den süßen Frühlingsmorgen}

\author{Ekkart Kleinod}
\institute[Heiterkeitsrunde]{Heiterkeitsrunde im trauten Heim}
\date{28. November 2011}


% Thema der Dokumenteigenschaften
\subject{Heiterkeit, Frühlingsmorgen}

% Folgendes sollte gelöscht werden, wenn man nicht am Anfang jedes
% Unterabschnitts die Gliederung nochmal sehen möchte.
%\AtBeginSubsection[]
%{
%	\begin{frame}<beamer>{Agenda}
%		\tableofcontents[currentsection,currentsubsection]
%	\end{frame}
%}

\AtBeginSection[]
{
	\begin{frame}<beamer>{Agenda}
		\tableofcontents[currentsection, hideallsubsections]
	\end{frame}
}


\begin{document}

	\mode<article> {
		\maketitle
	}

	\begin{frame}
		\titlepage
	\end{frame}

	\section*{Agenda}

	\begin{frame}{Agenda}
		\tableofcontents[hideallsubsections]
		% Die Option [pausesections] könnte nützlich sein.
	\end{frame}


	\section{Fließtext}

		\subsection{Heiterkeit}

		\begin{frame}{\insertsectionhead}{\insertsubsectionhead}
			Eine wunderbare Heiterkeit hat meine ganze Seele eingenommen, gleich den süßen Frühlingsmorgen, die ich mit ganzem Herzen genieße.
			Ich bin allein und freue mich meines Lebens in dieser Gegend, die für solche Seelen geschaffen ist wie die meine.

			Ich bin so glücklich, mein Bester, so ganz in dem Gefühle von ruhigem Dasein versunken, daß meine Kunst darunter leidet.
			Ich könnte jetzt nicht zeichnen, nicht einen Strich, und bin nie ein größerer Maler gewesen als in diesen Augenblicken.
		\end{frame}

		\subsection{Das Tal}

		\begin{frame}{\insertsectionhead}{\insertsubsectionhead}
			Wenn das liebe Tal um mich dampft, und die hohe Sonne an der Oberfläche der undurchdringlichen Finsternis meines Waldes ruht, und nur einzelne Strahlen sich in das innere Heiligtum stehlen, ich dann im hohen Grase am fallenden Bache liege, und näher an der Erde tausend mannigfaltige Gräschen mir merkwürdig werden; wenn ich das Wimmeln der kleinen Welt zwischen Halmen, die unzähligen, unergründlichen Gestalten der Würmchen, der Mückchen näher an meinem Herzen fühle, und fühle die Gegenwart des Allmächtigen, der uns nach seinem Bilde schuf, das Wehen des Alliebenden, der uns in ewiger Wonne schwebend trägt und erhält; mein Freund!
		\end{frame}


	\section{Grafiken}

		\subsection{Eine TikZ-Grafik}

		\begin{frame}{\insertsectionhead}{\insertsubsectionhead}
			\begin{tikzpicture}
				\tiny
				\ttfamily
				\colorlet{good}{green!75!black}
				\colorlet{bad}{red}
				\colorlet{neutral}{black!60}
				\colorlet{none}{white}

				\node[align=center,text width=3cm]{Ratings given by 50~participants};

				\begin{scope}[line width=4mm,rotate=270]
					\draw[good]          (-123:2cm) arc (-123:-101:2cm);
					\draw[good!60!white] (-36:2cm) arc (-36:-101:2cm);
					\draw[neutral]       (-36:2cm) arc (-36:36:2cm);
					\draw[bad!60!white]  (36:2cm)  arc (36:93:2cm);

					\newcount\mycount
					\foreach \angle in {0,72,...,3599}
					{
						\mycount=\angle\relax
						\divide\mycount by 10\relax
						\draw[black!15,thick] (\the\mycount:18mm) -- (\the\mycount:22mm);
					}

					\draw (0:2.2cm) node[below] {``ok'': 10 (20\%)};
					\draw (165:2.2cm) node[above] {none: 20 (40\%)};
					\draw (-111:2.2cm) node[left] {``very good'': 3 (6\%)};
					\draw (-68:2.2cm) node[left] {``good'': 9 (18\%)};
					\draw (65:2.2cm) node[right] {``bad'': 8 (16\%)};
					\draw (93:2.2cm) node[right] {``very bad'': 0 (0\%)};
				\end{scope}
				\draw[gray] (0,0) circle (2.2cm) circle (1.8cm);
			\end{tikzpicture}
		\end{frame}

		\subsection{Eine PDF-Grafik unskaliert}

		\begin{frame}{\insertsectionhead}{\insertsubsectionhead}
			\includegraphics{logo.pdf}
		\end{frame}

		\subsection{Eine PDF-Grafik skaliert auf 0.5 Textbreite}

		\begin{frame}{\insertsectionhead}{\insertsubsectionhead}
			\includegraphics[width=0.5\textwidth]{logo.pdf}
		\end{frame}

		\subsection{Eine PDF-Grafik skaliert auf 0.5 Texthöhe}

		\begin{frame}{\insertsectionhead}{\insertsubsectionhead}
			\includegraphics[height=0.5\textheight]{logo.pdf}
		\end{frame}



	\section{Aufzählungen}

		\subsection{Nicht numeriert}

		\begin{frame}{\insertsectionhead}{\insertsubsectionhead}
			\begin{itemize}
				\item Eine wunderbare Heiterkeit\dots
				\item Ich bin allein\dots
				\item Ich bin so glücklich, mein Bester\dots
				\item Ich könnte jetzt nicht zeichnen\dots
			\end{itemize}
		\end{frame}

		\subsection{Numeriert}

		\begin{frame}{\insertsectionhead}{\insertsubsectionhead}
			\begin{enumerate}
				\item Eine wunderbare Heiterkeit\dots
				\item Ich bin allein\dots
				\item Ich bin so glücklich, mein Bester\dots
				\item Ich könnte jetzt nicht zeichnen\dots
			\end{enumerate}
		\end{frame}

		\subsection{Beschreibung}

		\begin{frame}{\insertsectionhead}{\insertsubsectionhead}
			\begin{description}
				\item[Heiterkeit] Eine wunderbare Heiterkeit\dots
				\item[Allein] Ich bin allein\dots
				\item[Glücklich] Ich bin so glücklich, mein Bester\dots
				\item[Zeichnen] Ich könnte jetzt nicht zeichnen\dots
			\end{description}
		\end{frame}


	\section{Schrift}

		\subsection{Textauszeichnung}

		\begin{frame}{\insertsectionhead}{\insertsubsectionhead}
			\begin{description}
				\item[normal] Eine wunderbare Heiterkeit\dots
				\item[slanted] \textsl{Eine wunderbare Heiterkeit\dots}
				\item[bold] \textbf{Eine wunderbare Heiterkeit\dots}
				\item[italic] \textit{Eine wunderbare Heiterkeit\dots}
				\item[mono] \texttt{Eine wunderbare Heiterkeit\dots}
				\item[math] $\sqrt[4]{\sin a \cdot \alpha^2}$
			\end{description}
		\end{frame}

\end{document}

