\documentclass[t, aspectratio=169, ngerman]{beamer}
%\documentclass[aspectratio=169, notes=show, ngerman]{beamer}
%\documentclass[aspectratio=169, handout, ngerman]{beamer}
%\documentclass[aspectratio=169, handout, notes=show, ngerman]{beamer}
%\documentclass[aspectratio=169, handout, notes=only, ngerman]{beamer}

% t: Texte oben ausgerichtet
% aspectratio: 169, 1610, 149, 54, 32, 43

%\usetheme{basis}
\usetheme[titleprogressbar]{basis}
%\usetheme[totalnumber]{basis}
%\usetheme[nonumber]{basis}
%\usetheme[nonumber,noauthor]{basis}

\usepackage[pangram]{blindtext}

\subject{Thema}
\title{Titel}
\subtitle{Untertitel}
\author{Autor 1 \and Autor 2 \and Autor 3}
\date{\today}

\logo{\includegraphics[width=5em]{testlogo}}

\begin{document}

	\maketitle

	\begin{frame}
		Nicht vergessen: mit \hologo{pdfLaTeX} oder \hologo{XeLaTeX} übersetzen.

		Diese Präsentation dient zur Überprüfung des Aussehens und dessen Definition, nicht als Beispiel.
	\end{frame}

	\begin{frame}
		\frametitle{Agenda}
		\tableofcontents
	\end{frame}

	\section{Normaler Text}

	\begin{frame}{Text}
		\Blindtext[1][4]
	\end{frame}

	\begin{frame}{Langer Text}
		\Blindtext[2][4]
	\end{frame}

	\begin{frame}{Text mit Auszeichnungen und Fußnoten}
		Prall vom \emph{Whisky}\footnote{emph} flog \textbf{Quax}\footnote{textbf} den Jet zu Bruch.
		Jeder \textit{wackere}\footnote{textit} Bayer \textsl{vertilgt}\footnote{textsl} bequem zwo Pfund Kalbshaxen.
		Stanleys \alert{Expeditionszug}\footnote{alert} quer durch Afrika wird von jedermann bewundert.
		Zwölf Boxkämpfer \texttt{jagen Viktor}\footnote{texttt} quer über den großen Sylter Deich.
	\end{frame}

	\section{Aufzählungen}

	\begin{frame}{Unnumeriert}
		\blindlistlist[1]{itemize}[3]
	\end{frame}

	\begin{frame}{Unnumeriert}
		\blindlistlist[3]{itemize}[3]
	\end{frame}

	\begin{frame}{Unnumeriert mit langem Text}
		\Blindlist{itemize}[2]
	\end{frame}

	\begin{frame}{Numeriert}
		\blindlistlist[1]{enumerate}[3]
	\end{frame}

	\begin{frame}{Numeriert}
		\blindlistlist[3]{enumerate}[3]
	\end{frame}

	\begin{frame}{Beschreibung}
		\blindlistoptional{description}[3]
	\end{frame}

	\section{Grafiken}

	\begin{frame}{Eingebundenes PDF}
		\includegraphics{testlogo}
	\end{frame}

	\begin{frame}{Eingebundenes PDF als Figure (1)}
		\begin{figure}
			\includegraphics{testlogo}
			\caption{Bildunterschrift unter Bild.}
		\end{figure}
		\note[item]{Anmerkung dazu.}
	\end{frame}

	\begin{frame}{Eingebundenes PDF als Figure (2)}
		\begin{figure}
			\caption{Bildunterschrift über Bild.}
			\includegraphics{testlogo}
		\end{figure}
	\end{frame}

	\section{Mathematik}

	\begin{frame}{Formeln im Fließtext}
		\blindmathtrue
		\Blindtext[1][5]
		\blindmathfalse
	\end{frame}

	\begin{frame}{Einzelne Formeln}
		\begin{align}
			a^2 + b^2 &= c^2\\
			(a+b)^2 &= a^2 + 2ab + b^2
		\end{align}
	\end{frame}

	\begin{frame}[plain]{Ende}
		\Huge{Ende.}
	\end{frame}

\end{document}

