\documentclass[t, aspectratio=169, ngerman]{beamer}
%\documentclass[t, aspectratio=169, handout, ngerman]{beamer}
%\documentclass[t, aspectratio=169, handout, notes=show, ngerman]{beamer}
%\documentclass[t, aspectratio=169, handout, notes=only, ngerman]{beamer}

% t: Texte oben ausgerichtet
% aspectratio: 169, 1610, 149, 54, 32, 43

\usetheme{basis}

\usepackage[pangram]{blindtext}

\subject{Thema}
\title{Titel}
\subtitle{Untertitel}
\author{Autor 1 \and Autor 2 \and Autor 3}
\date{6. Januar 2016}

\begin{document}

	\frame{\titlepage}

	\begin{frame}
		Nicht vergessen: mit \hologo{XeLaTeX} übersetzen.

		Diese Datei zeigt die verschiedenen Möglichkeiten mit Beamer.
	\end{frame}

	\section*{Agenda}

	\begin{frame}
		\frametitle{Agenda}
		\tableofcontents
	\end{frame}

	\begin{frame}
		\frametitle{Agenda zum Aufklappen}
		\tableofcontents[pausesections]
	\end{frame}

	\section{Normaler Text}

	\begin{frame}{Text}
		Franz jagt im komplett verwahrlosten Taxi quer durch Bayern.
		Zwölf Boxkämpfer jagen Viktor quer über den großen Sylter Deich.
		Vogel Quax zwickt Johnys Pferd Bim.
		Sylvia wagt quick den Jux bei Pforzheim.
	\end{frame}

	\begin{frame}{Langer Text}
		Prall vom Whisky flog Quax den Jet zu Bruch.
		Jeder wackere Bayer vertilgt bequem zwo Pfund Kalbshaxen.
		Stanleys Expeditionszug quer durch Afrika wird von jedermann bewundert.
		Franz jagt im komplett verwahrlosten Taxi quer durch Bayern.

		Zwölf Boxkämpfer jagen Viktor quer über den großen Sylter Deich.
		Vogel Quax zwickt Johnys Pferd Bim.
		Sylvia wagt quick den Jux bei Pforzheim.
		Prall vom Whisky flog Quax den Jet zu Bruch.
	\end{frame}

	\begin{frame}{Text mit Auszeichnungen und Fußnoten}
		Prall vom \emph{Whisky}\footnote{emph} flog \textbf{Quax}\footnote{textbf} den Jet zu Bruch.
		Jeder \textit{wackere}\footnote{textit} Bayer \textsl{vertilgt}\footnote{textsl} bequem zwo Pfund Kalbshaxen.
		Stanleys \alert{Expeditionszug}\footnote{alert} quer durch Afrika wird von jedermann bewundert.
	\end{frame}

	\begin{frame}{Ein Zitat (quote)}
		Prall vom Whisky flog Quax den Jet zu Bruch.

		\begin{quote}
			Jeder wackere Bayer vertilgt bequem zwo Pfund Kalbshaxen.

			Stanleys Expeditionszug quer durch Afrika wird von jedermann bewundert.
			Franz jagt im komplett verwahrlosten Taxi quer durch Bayern.
		\end{quote}

		Zwölf Boxkämpfer jagen Viktor quer über den großen Sylter Deich.
	\end{frame}

	\begin{frame}{Ein Zitat (quotation)}
		Prall vom Whisky flog Quax den Jet zu Bruch.

		\begin{quotation}
			Jeder wackere Bayer vertilgt bequem zwo Pfund Kalbshaxen.

			Stanleys Expeditionszug quer durch Afrika wird von jedermann bewundert.
			Franz jagt im komplett verwahrlosten Taxi quer durch Bayern.
		\end{quotation}

		Zwölf Boxkämpfer jagen Viktor quer über den großen Sylter Deich.
	\end{frame}

	\section{Aufzählungen}

	\subsection{Aufzählungen ohne Nummern}

	\begin{frame}{Unnumeriert}
		\begin{itemize}
			\item Erster Listenpunkt, Stufe 1
			\item Zweiter Listenpunkt, Stufe 1
			\item Dritter Listenpunkt, Stufe 1
		\end{itemize}
	\end{frame}

	\begin{frame}{Unnumeriert mit Pause}
		\begin{itemize}
			\item Erster Listenpunkt, Stufe 1
				\pause
			\item Zweiter Listenpunkt, Stufe 1
				\pause
			\item Dritter Listenpunkt, Stufe 1
		\end{itemize}
	\end{frame}

	\subsection{Aufzählungen mit Nummern}

	\begin{frame}{Numeriert}
		\begin{enumerate}
			\item Erster Listenpunkt, Stufe 1
			\item Zweiter Listenpunkt, Stufe 1
			\item Dritter Listenpunkt, Stufe 1
		\end{enumerate}
	\end{frame}

	\subsection{Aufzählungen mit Beschreibungstexten}

	\begin{frame}{Beschreibung}
		\begin{description}
			\item[Erster] Listenpunkt, Stufe 1
			\item[Zweiter] Listenpunkt, Stufe 1
			\item[Dritter] Listenpunkt, Stufe 1
		\end{description}
	\end{frame}

	\section{Grafiken}

	\begin{frame}{Eingebundenes PDF}
		\includegraphics{testlogo}
	\end{frame}

	\begin{frame}{Eingebundenes PDF skaliert}
		\includegraphics[width=\textwidth]{testlogo}
	\end{frame}

	\section{Spezielles}

	\begin{frame}{Definition und Beispiel}
		\begin{definition}
			Hier ist eine Definition.
		\end{definition}
		\begin{example}
			Hier ist ein Beispiel dazu.
		\end{example}
	\end{frame}

	\begin{frame}{Satz und Beweis mit Aufblättern}
		\begin{theorem}
			Eine Behauptung.
		\end{theorem}
		\begin{proof}
			\begin{enumerate}
				\item<1-> Sei es so.
				\item<2-> Dann folgt dies daraus.
				\item<1-> Damit gilt das immer.\qedhere
			\end{enumerate}
		\end{proof}
		\uncover<3->{Ui.}
	\end{frame}

	\begin{frame}{Horizontale Blöcke}
		\begin{block}{Block 1}
			Inhalt des ersten Blocks.
		\end{block}
		\begin{block}{Block 2}
			Inhalt des zweiten Blocks.

			Auch Inhalt des zweiten Blocks.
		\end{block}
	\end{frame}

	\begin{frame}{Vertikale Blöcke (Spalten)}
		\begin{columns}
			\column{.3\textwidth}
				\begin{block}{Block 1}
					Inhalt des ersten Blocks.
				\end{block}
			\column{.7\textwidth}
				\begin{block}{Block 2}
					Inhalt des zweiten Blocks.

					Auch Inhalt des zweiten Blocks.
				\end{block}
		\end{columns}
	\end{frame}

	\begin{frame}[fragile]{Code und anderer Verbatimtext.}
		\begin{verbatim}
			function returnNull() {
				return null;
			}
		\end{verbatim}
		\begin{uncoverenv}<2>
			Note the use of \verb|null|.
		\end{uncoverenv}
	\end{frame}

	\begin{frame}[label=frame1]{Label und Referenz auf sich selbst}
		Hallo, ich bin Frame~\autoref{frame1}.
	\end{frame}

	\begin{frame}[plain]{Ende}
		\Huge{Ende.}
	\end{frame}

\end{document}

