\documentclass[ngerman]{beamer}
%\documentclass[handout, ngerman]{beamer}
%\documentclass[handout, notes=show, ngerman]{beamer}
%\documentclass[handout, notes=only, ngerman]{beamer}

\usetheme{basis}

\usepackage[pangram]{blindtext}
\renewcommand{\blindmarkup}[1]{\emph{#1}}

\subject{Thema}
\title{Titel}
\subtitle{Untertitel}
\author{Autor 1 \and Autor 2 \and Autor 3}
\date{6. Januar 2016}

\begin{document}

	\frame{\titlepage}

	\begin{frame}
		Nicht vergessen: mit \hologo{XeLaTeX} übersetzen.
	\end{frame}

	\begin{frame}
		\frametitle{Agenda}
		\tableofcontents
	\end{frame}

	\section{Normaler Text}

	\begin{frame}{Text}
		\Blindtext[1][4]
	\end{frame}

	\begin{frame}{Langer Text}
		\Blindtext[2][4]
	\end{frame}

	\begin{frame}{Text mit Auszeichnungen}
		\Blindtext[1][4]
	\end{frame}

	\section{Aufzählungen}

	\begin{frame}{Unnumeriert}
		\blindlistlist[1]{itemize}[3]
	\end{frame}

	\begin{frame}{Unnumeriert}
		\blindlistlist[3]{itemize}[3]
	\end{frame}

	\begin{frame}{Unnumeriert mit langem Text}
		\Blindlist{itemize}[2]
	\end{frame}

	\begin{frame}{Numeriert}
		\blindlistlist[1]{enumerate}[3]
	\end{frame}

	\begin{frame}{Numeriert}
		\blindlistlist[3]{enumerate}[3]
	\end{frame}

	\begin{frame}{Beschreibung}
		\blindlistoptional{description}[3]
	\end{frame}

	\section{Sonstiges}

	\begin{frame}{Formeln im Fließtext}
		\blindmathtrue
		\Blindtext[1][5]
		\blindmathfalse
	\end{frame}

	\begin{frame}
		\Huge{Ende.}
	\end{frame}

\end{document}

