%^^A Nutzerdokumentation

\section{Einleitung}

Dieses Paket dient dazu, einen Basis-Stil zu definieren, der Dokumente und Briefe setzt und dabei alle benötigten Pakete lädt und initialisiert.

Das Paket ist zum privaten Einsatz gedacht, wer es nutzen will, sei herzlich dazu eingeladen, die Weitergabe sollte vollständig erfolgen, eigene Änderungen sollten als solche gekennzeichnet werden.

Ein offenes Ohr für Verbesserungsvorschläge oder Kritik habe ich per E-Mail immer, Zeit nicht so oft \smiley

Offizielle Lizenz ist Creative-Commons-Lizenz Namensnennung - Nicht-kommerziell - Weitergabe unter gleichen Bedingungen 4.0 International (\url{http://creativecommons.org/licenses/by-nc-sa/4.0/deed.de}).


\cleardoublepage
\thispagestyle{BASfuss}
\part{Nutzerinformationen}
\cleardoublepage


\section{Die Dateien}
\label{sec:Dateien}

Folgende Dateien gehören zum Basis-Stil:
{\small
\begin{verbatim}
/source/latex/basis
	basis.drv
	basis.dtx
	basis.ins
	basis.userdoc.tex
	build.xml
/texmf
	/doc/latex/basis
		basis.pdf
		basis_short.pdf
	/doc/latex/basis/vorlagen
		brief.mmd
		brief.tex
		dokument.mmd
		dokument.tex
	/tex/latex/basis
		basbrief.sty
		basinfospaltefett.lco
		basinfospalte.lco
		basis-mmd-begin-doc.tex
		basis-mmd-begin-lttr.tex
		basis-mmd-end-lttr.tex
		basis-mmd-scrartcl.tex
		basis-mmd-scrbook.tex
		basis-mmd-scrlttr2.tex
		basis-mmd-scrreprt.tex
		basis-mmd-style.tex
		basis.sty
		baskopfzeile.lco
		beamercolorthemebasis.sty
		beamerfontthemebasis.sty
		beamerinnerthemebasis.sty
		beamerouterthemebasis.sty
		beamerthemebasis.sty
\end{verbatim}
}

\minisec{Dokumentation}

Die Datei \datei{basis\_short.pdf} enthält die Nutzerdokumentation des Basis-Pakets.
Die Datei \datei{basis.pdf} enthält die Nutzerdokumentation und den dokumentierten Quellcode des Basis-Pakets.

\minisec{Vorlagen}

Die Vorlagen sind \hologo{LaTeX}- und Multimarkdown-Dateien, die für eigene Dokumente genutzt werden können.
Sie sind einfach in das eigene Verzeichnis zu kopieren, anzupassen und fertig.

\minisec{Stildateien}

Die Stildateien sind die Dateien, die beim \hologo{LaTeX}-Lauf zur Formatierung genutzt werden.

\minisec{Quelltext}

Der \datei{source}-Zweig enthält den Quelltext des Basis-Pakets.
Alle Änderungen sind hier vorzunehmen und die anderen Dateien zu generieren.



\section{Die Benutzerschnittstelle des Basis-Stils}

\subsection{Benutzung}

Das \paket{basis}-Paket basiert auf den KOMA-Script-Klassen.
Daher sind diese als Dokumentklasse für Dokumente zu laden, bevor das \paket{basis}-Paket eingebunden wird.
Um einen Text im Basis-Layout zu setzen, ist es notwendig, das \paket{basis}-Paket wie folgt zu benutzen:

\begin{nutzung}
		\>\befehl{usepackage[\meta{Optionen}]\{basis\}}\\
	\beispiel
		\>\befehl{documentclass[ngerman]\{scrartcl|scrbook|scrreprt\}}\\
		\>\befehl{usepackage\{basis\}}\\
		\>\befehl{begin\{document\}}\\
		\>\texttt{\dots}\\
		\>\befehl{end\{document\}}\\
\end{nutzung}

Dem Paket können Optionen übergeben werden, die in \autoref{sec:userdoc:options} erläutert werden.

\subsection{Benötigte Pakete}
\label{sec:userdoc:packages}

Das \paket{basis}-Paket bindet die Pakete bereits ein, die entweder für das Paket notwendig sind oder für das Schreiben hilfreich sind.
Die Pakete werden im folgenden kurz vorgestellt und müssen für die Nutzung des \paket{basis}-Pakets zur Verfügung stehen.
Eine genauere Beschreibung der einzelnen Pakete ist in der Dokumentation der Pakete selbst zu finden.

Die Pakete müssen vom Anwender auf dem Rechner installiert werden, sonst gibt es Fehlermeldungen.

Die Pakete sind (alphabetisch sortiert):
\begin{tabbing}
	\paket{marvosym}mm\=\kill

	\paket{array}\>Tabellenerweiterung\\
	\paket{babel}\>Das Sprachpaket von \LaTeX\\
	\paket{booktabs}\>schöne Tabellenlinien\\
	\paket{dhua}\>Eingabe gebräuchlicher Abkürzungen\\
	\paket{enumitem}\>einfachere Optionen für Aufzählungen und Listen\\
	\paket{fontenc}\>erweitertes Font-Encoding\\
	\paket{graphicx}\>Ausgabe von Grafiken\\
	\paket{hologo}\>\hologo{LaTeX}-Logos (wenn Paket fehlt, wird eine Warnung ausgegeben)\\
	\paket{hyperref}\>Inhaltsverzeichnis und navigierbare Links\\
	\paket{ifpdf}\>Abfrage, ob \hologo{pdfLaTeX} zur Übersetzung genutzt wird\\
	\paket{ifxetex}\>Abfrage, ob \hologo{XeTeX} zur Übersetzung genutzt wird\\
	\paket{inputenc}\>direkte Eingabe von Sonderzeichen im laufenden Text\\
	\paket{lastpage}\>Seitenzähler\\
	\paket{longtable}\>lange Tabellen\\
	\paket{marvosym}\>Sonderzeichen (wenn Paket fehlt, wird eine Warnung ausgegeben)\\
	\paket{microtype}\>ausgeglichenerer Schrriftsatz incl.\ Randausgleich\\
	\paket{pdfcolmk}\>Problemlösung bei Textfärbung\\
	\paket{ragged2e}\>verbesserter Flattersatz\\
	\paket{scrpage2}\>selbst definierte Kopf- bzw.\ Fußzeilen\\
	\paket{tabu}\>Tabellenerweiterung\\
	\paket{wasysym}\>Sonderzeichen (wenn Paket fehlt, wird eine Warnung ausgegeben)\\
	\paket{xcolor}\>Textfärbung\\
	\paket{xifthen}\>vereinfachte if-then-Abfragen\\
	\paket{xkeyval}\>key-value-Optionen
\end{tabbing}

Zusätzlich werden, je nach gewählter Font-Option, bestimmte Schriftartenpakete geladen.
\begin{tabbing}
	\option{font=mathpazo}mm\=\kill

	\option{font=charter}\>\paket{charter}, \paket{helvet}, \paket{luximono}\\
	\option{font=droid}\>\paket{droid}\\
	\option{font=hfold}\>\paket{hfoldsty}\\
	\option{font=mathpazo}\>\paket{mathpazo}, \paket{helvet}, \paket{luximono}\\
	\option{font=original}\>keine Zusatzklassen\\
	\option{font=times}\>\paket{mathptmx}, \paket{helvet}, \paket{luximono}
\end{tabbing}


\subsection{Optionen}
\label{sec:userdoc:options}

Die Optionen des \paket{basis}-Pakets werden einzeln erläutert, sie können auch beliebig kombiniert werden, die geschieht durch Trennung mit Kommata.

\begin{nutzung}
		\>\befehl{usepackage[\meta{Optionen}]\{basis\}}\\
	\beispiel
		\>\befehl{usepackage[draft]\{basis\}}\\
		\>\befehl{usepackage[draft, font=charter]\{basis\}}
\end{nutzung}

Mögliche Optionen: \option{draft}, \option{final}, \option{font}, \option{fontsize}, \option{hypercolor}, \option{hyperdriver}, \option{layout}, \option{nobackaddress}, \option{nofoldmarks}, \option{notitlepage}, \option{oneside}, \option{protocol}, \option{pagestyle}

\subsubsection{draft}

Die \option{draft}-Option setzt das Dokument als Entwurfsdokument.
Die Option wird lediglich an die Dokumentklasse sowie Pakete \paket{graphicx} und \paket{hyperref} weitergegeben.

\begin{nutzung}
		\>\befehl{usepackage[draft]\{basis\}}\\
\end{nutzung}

\subsubsection{final}

Die \option{final}-Option setzt das Dokument als fertiges Dokument.
Die Option wird lediglich an die Pakete \paket{graphicx} und \paket{hyperref} weitergegeben.

\begin{nutzung}
		\>\befehl{usepackage[final]\{basis\}}\\
\end{nutzung}

\subsubsection{font}

Die \option{font}-Option sorgt für die Einstellung eines bestimmten Fontschemas.
Die konkret geladenen Schriften sind in \autoref{sec:userdoc:packages} dokumentiert.

Standard: \option{times}

\begin{nutzung}
		\>\befehl{usepackage[font=\meta{charter|hfold|mathpazo|original|times}]\{basis\}}\\
	\beispiel
		\>\befehl{usepackage[font=charter]\{basis\}}\\
		\>\befehl{usepackage[font=times]\{basis\}}
\end{nutzung}

\subsubsection{fontsize}

Die \option{fontsize}-Option legt die Schriftgröße des Dokuments fest.

Standard: \option{11pt}

\begin{nutzung}
		\>\befehl{usepackage[fontsize=\meta{Größe}]\{basis\}}\\
	\beispiel
		\>\befehl{usepackage[fontsize=10pt]\{basis\}}\\
		\>\befehl{usepackage[fontsize=1cm]\{basis\}}
\end{nutzung}

\subsubsection{hypercolor}

Die Option \option{hypercolor} färbt Referenzen in der gewünschten Farbe.
Die verfügbaren Farbnamen sind in der Dokumentation des Pakets \paket{xcolor} zu finden.

Standard: \option{schwarz}

\begin{nutzung}
		\>\befehl{usepackage[hypercolor=\meta{Farbe}]\{basis\}}\\
	\beispiel
		\>\befehl{usepackage[hypercolor=blue]\{basis\}}\\
		\>\befehl{usepackage[hypercolor=green]\{basis\}}
\end{nutzung}

\subsubsection{hyperdriver}

Die Option \option{hyperdriver} setzt den durch \paket{hyperref} genutzten Treiber.

Standard: \hologo{LaTeX}: \option{ps2pdf}; \hologo{pdfLaTeX}: \option{pdftex}; \hologo{XeTeX}: \option{xetex}

\begin{nutzung}
		\>\befehl{usepackage[hyperdriver=\meta{Treibername}]\{basis\}}\\
	\beispiel
		\>\befehl{usepackage[hyperdriver=dvips]\{basis\}}
\end{nutzung}


\subsubsection{layout}

Die \option{layout}-Option sorgt für die Einstellung eines bestimmten Brieflayouts.
Die Option wird nur bei Briefen ausgewertet.

Standard: \option{infospalte}

\begin{nutzung}
		\>\befehl{usepackage[layout=\meta{kopfzeile|infospalte|infospaltefett}]\{basis\}}\\
	\beispiel
		\>\befehl{usepackage[layout=kopfzeile]\{basis\}}\\
		\>\befehl{usepackage[layout=infospaltefett]\{basis\}}
\end{nutzung}

\begin{tabbing}
	\option{infospaltefett}mm\=\kill

	\option{kopfzeile}\>Adressangaben in Kopfzeile\\
	\option{infospalte}\>Adressangaben in separater Spalte, Autor oben in Kapitälchen\\
	\option{infospaltefett}\>wie \option{infospalte}, Autor oben in Fettdruck
\end{tabbing}

\subsubsection{nobackaddress}

Die \option{nobackaddress}-Option verhindert die Anzeige der Rücksendeadresse im Adressfeld.
Die Option wird nur bei Briefen ausgewertet.

\begin{nutzung}
		\>\befehl{usepackage[nobackaddress]\{basis\}}\\
\end{nutzung}

\subsubsection{nofoldmarks}

Die \option{nofoldmarks}-Option verhindert die Anzeige der Falzmarken.
Die Option wird nur bei Briefen ausgewertet.

\begin{nutzung}
		\>\befehl{usepackage[nofoldmarks]\{basis\}}\\
\end{nutzung}

\subsubsection{notitlepage}

Die \option{notitlepage}-Option unterdrückt eine separate Titelseite und setzt stattdessen eine laufende Titelseite im Fließtext.
Die Option wird nur bei Artikeln oder Büchern ausgewertet.
Die Titelseite muss wie üblich im Text mit \befehl{titlepage} gesetzt werden.

\begin{nutzung}
		\>\befehl{usepackage[notitlepage]\{basis\}}\\
\end{nutzung}

\subsubsection{oneside}

Die \option{oneside}-Option setzt den Text einseitig,

\begin{nutzung}
		\>\befehl{usepackage[oneside]\{basis\}}\\
\end{nutzung}

\subsubsection{protocol}

Die \option{protocol}-Option nimmt Änderungen für Protokolle vor.
Die Option wird nur bei Artikeln oder Büchern ausgewertet.

Die Änderungen sind:
\begin{itemize}
	\item Inhaltsverzeichnis heißt "`Tagesordnung"'
\end{itemize}

\begin{nutzung}
		\>\befehl{usepackage[protocol]\{basis\}}\\
\end{nutzung}



\subsection{Neue bzw.\ geänderte Befehle und Umgebungen}

\subsubsection{Dokumentinformationen}

\DescribeMacro{\title}
Angabe des Dokumenttitels.

Optionales Argument: Kurztitel.

Nutzung:
\begin{itemize}
	\item Titelseite (Artikel, Bücher)
	\item Fußzeile (Kurztitel, wenn angegeben) (Artikel, Bücher)
	\item Dokumenteigenschaften (Artikel, Bücher, Briefe)
\end{itemize}

\begin{nutzung}
		\>\befehl{title[\meta{Kurztitel}]\{\meta{Titel}\}}\\
	\beispiel
		\>\befehl{title\{Das basis-Paket\}}\\
		\>\befehl{title[basis-Paket]\{Das basis-Paket\}}
\end{nutzung}

\DescribeMacro{\subtitle}
Angabe des Dokumentuntertitels.

Nutzung:
\begin{itemize}
	\item Titelseite (Artikel, Bücher)
	\item Dokumenteigenschaften (Artikel, Bücher, Briefe)
\end{itemize}

\begin{nutzung}
		\>\befehl{subtitle\{\meta{Untertitel}\}}\\
	\beispiel
		\>\befehl{subtitle\{Ein LaTeX-Stil mit Basisanpassungen\}}
\end{nutzung}

\DescribeMacro{\version}
Angabe der Dokumentversion.

Nutzung:
\begin{itemize}
	\item Titelseite (Artikel, Bücher)
\end{itemize}

\begin{nutzung}
		\>\befehl{version\{\meta{Version}\}}\\
	\beispiel
		\>\befehl{version\{Version 0.4\}}
\end{nutzung}

\DescribeMacro{\date}
Angabe des Dokumentdatums.

Nutzung:
\begin{itemize}
	\item Titelseite (Artikel, Bücher)
\end{itemize}

\begin{nutzung}
		\>\befehl{date\{\meta{Datum}\}}\\
	\beispiel
		\>\befehl{date\{16. Dezember 2013\}}\\
		\>\befehl{date\{\befehl{today}\}}
\end{nutzung}

\DescribeMacro{\author}
Angabe des Autors.

Nutzung:
\begin{itemize}
	\item Titelseite (Artikel, Bücher)
	\item Dokumenteigenschaften (Artikel, Bücher, Briefe)
	\item Absendername (Briefe)
\end{itemize}

\begin{nutzung}
		\>\befehl{author\{\meta{Autor}\}}\\
	\beispiel
		\>\befehl{author\{Ekkart Kleinod\}}
\end{nutzung}

\DescribeMacro{\briefkopf}
Angabe eines separaten Briefkopfs, wenn der Autor nicht genutzt werden soll.

Nutzung:
\begin{itemize}
	\item Briefkopf (Briefe)
\end{itemize}

\begin{nutzung}
		\>\befehl{briefkopf\{\meta{Briefkopf}\}}\\
	\beispiel
		\>\befehl{briefkopf\{edgesoft\}}
\end{nutzung}

\DescribeMacro{\strasse}
Angabe der Straße des Autors.

Nutzung:
\begin{itemize}
	\item Titelseite (Artikel, Bücher)
	\item Absenderinformationen (Briefe)
\end{itemize}

\begin{nutzung}
		\>\befehl{strasse\{\meta{Straße}\}}\\
	\beispiel
		\>\befehl{strasse\{Musterstraße 23\}}
\end{nutzung}

\DescribeMacro{\plz}
Angabe der PLZ des Autors.

Nutzung:
\begin{itemize}
	\item Titelseite (Artikel, Bücher)
	\item Absenderinformationen (Briefe)
\end{itemize}

\begin{nutzung}
		\>\befehl{plz\{\meta{PLZ}\}}\\
	\beispiel
		\>\befehl{plz\{10001\}}
\end{nutzung}

\DescribeMacro{\ort}
Angabe des Orts des Autors.

Nutzung:
\begin{itemize}
	\item Titelseite (Artikel, Bücher)
	\item Absenderinformationen (Briefe)
\end{itemize}

\begin{nutzung}
		\>\befehl{ort\{\meta{Ort}\}}\\
	\beispiel
		\>\befehl{ort\{Musterort\}}
\end{nutzung}

\DescribeMacro{\telefon}
Angabe der Telefonnummer des Autors.

Nutzung:
\begin{itemize}
	\item Titelseite (Artikel, Bücher)
	\item Absenderinformationen (Briefe)
\end{itemize}

\begin{nutzung}
		\>\befehl{telefon\{\meta{Telefonnummer}\}}\\
	\beispiel
		\>\befehl{telefon\{030 123456\}}
\end{nutzung}

\DescribeMacro{\handy}
Angabe der Handynummer des Autors.

Nutzung:
\begin{itemize}
	\item Titelseite (Artikel, Bücher)
	\item Absenderinformationen (Briefe)
\end{itemize}

\begin{nutzung}
		\>\befehl{handy\{\meta{Handynummer}\}}\\
	\beispiel
		\>\befehl{handy\{0175 123456\}}
\end{nutzung}

\DescribeMacro{\email}
Angabe der E-Mail-Adresse des Autors.

Nutzung:
\begin{itemize}
	\item Titelseite (Artikel, Bücher)
	\item Absenderinformationen (Briefe)
\end{itemize}

\begin{nutzung}
		\>\befehl{email\{\meta{E-Mail-Adresse}\}}\\
	\beispiel
		\>\befehl{email\{ekleinod@edgesoft.de\}}
\end{nutzung}

\DescribeMacro{\homepage}
Angabe der Homepage des Autors.

Nutzung:
\begin{itemize}
	\item Titelseite (Artikel, Bücher)
	\item Absenderinformationen (Briefe)
\end{itemize}

\begin{nutzung}
		\>\befehl{homepage\{\meta{URL}\}}\\
	\beispiel
		\>\befehl{homepage\{https://github.com/ekleinod/basis\}}
\end{nutzung}

\DescribeMacro{\adresszusatz}
Angabe eines Adresszusatztexts.

Nutzung:
\begin{itemize}
	\item Titelseite (Artikel, Bücher)
	\item Absenderinformationen (Briefe)
\end{itemize}

\begin{nutzung}
		\>\befehl{adresszusatz\{\meta{Zusatztext}\}}\\
	\beispiel
		\>\befehl{adresszusatz\{Bevorzugt per E-Mail.\}}
\end{nutzung}

\DescribeMacro{\titelzusatz}
Angabe eines Titelzusatztexts.

Nutzung:
\begin{itemize}
	\item Titelseite (Artikel, Bücher)
\end{itemize}

\begin{nutzung}
		\>\befehl{titelzusatz\{\meta{Zusatztext}\}}\\
	\beispiel
		\>\befehl{titelzusatz\{Selbst geschrieben.\}}
\end{nutzung}

\DescribeMacro{\logo}
Angabe eines Logos (Dateiname ohne Endung).

Nutzung:
\begin{itemize}
	\item Titelseite (Artikel, Bücher)
	\item Kopfzeile (Briefe)
\end{itemize}

\begin{nutzung}
		\>\befehl{logo\{\meta{Dateiname}\}}\\
	\beispiel
		\>\befehl{logo\{testlogo\}}
\end{nutzung}


\subsubsection{Sonstiges}

\DescribeMacro{\meta}
Der Befehl \befehl{meta} setzt den übergebenen Text als \meta{Metatext}.
Das bedeutet, dass spitze Klammern um den schräg gestellten Text geschrieben werden.
\begin{nutzung}
		\>\befehl{meta\{\meta{Metatext}\}}\\
	\beispiel
		\>\befehl{meta\{Metatext\}}
\end{nutzung}

\subsubsection{Vortragsdokumentation}

Diese Befehle sind noch experimentell.

\DescribeMacro{\insertslide}
Der Befehl \befehl{insertslide} fügt das Bild einer Folie ein.
Genau gesagt, wird ein Bild rechtsseitig gerahmt mit einer anzugebenden Skalierung eingebunden.
Die Einbindung erfolgt über \befehl{includegraphics}, die Skalierungsangabe ist dementsprechend zu wählen.
Die Skalierung ist der erste Parameter, der Präfix des Bildnamens der zweite.

\begin{nutzung}
		\>\befehl{insertslide\{\meta{Skalierung}\}\{\meta{Präfix}\}}\\
	\beispiel
		\>\befehl{insertslide\{width=.3\befehl{textwidth}\}\{slide\}}\\
		\>\befehl{insertslide\{angle=45,width=.2\befehl{textwidth}\}\{img\}}
\end{nutzung}

\DescribeMacro{\nextslide}
Der Befehl \befehl{nextslide} kapselt den Aufruf von \befehl{insertslide} mit für OpenOffice"=Folien günstigen Werten.
Die Skalierung wird auf 30\,\% der Textbreite gesetzt, die Dateien müssen mit \emph{slide} beginnen.
Außerdem wird der Folienzähler um eins erhöht.

\begin{nutzung}
		\>\befehl{nextslide}
\end{nutzung}

\DescribeMacro{\nextslidesilent}
Der Befehl \befehl{nextslidesilent} erhöht den Folienzähler um eins, ohne die entsprechende Folie auszugeben.
Damit können \zB für die Dokumentation unwichtige Folien übersprungen werden.

\begin{nutzung}
		\>\befehl{nextslidesilent}
\end{nutzung}

%^^A Ende Nutzerdokumentation

