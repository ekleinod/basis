%^^A Nutzerdokumentation

\section{Einleitung}

Dieses Paket dient dazu, einen Basis-Stil zu definieren, der Dokumente und Briefe
setzt und dabei alle benötigten Pakete lädt und initialisiert.

Das Paket ist zum privaten Einsatz gedacht, wer es nutzen will, sei herzlich dazu
eingeladen, die Weitergabe sollte vollständig erfolgen, eigene Änderungen sollten
als solche gekennzeichnet werden.

Ein offenes Ohr für Verbesserungsvorschläge oder Kritik habe ich per Mail immer.


\cleardoublepage
\part{Nutzerinformationen}
\cleardoublepage


\section{Die Dateien}
\label{sec:Dateien}

Folgende Dateien gehören zum Basis-Stil:
{\small
\begin{verbatim}
/texmf
	/doc/latex/basis
		basis.pdf
		basis_short.pdf
	/doc/latex/basis/vorlagen
		artikel.tex
		brief.tex
	/makeindex/latex/basis
		basis.ist
	/source/latex/basis
		basis.drv
		basis.dtx
		basis.ins
		basis.tcp
		build.xml
		docstrip.cfg
	/tex/latex/basis
		basbrief.sty
		basis.sty
\end{verbatim}
}

\minisec{Dokumentation}

Die Datei \verb#basis_short.pdf# enthält die Nutzerdokumentation des Basis-Pakets.
Die Datei \verb#basis.pdf# enthält die Nutzerdokumentation und den Quellcode des Basis-Pakets.

\minisec{Vorlagen}

Die Vorlagen sind \TeX-Dateien, die für eigene Dokumente genutzt werden können.
Sie sind einfach in das eigene Verzeichnis zu kopieren, anzupassen und fertig.

\minisec{Makeindex-Stil}

Die Datei \verb#basis.ist# sorgt dafür, dass der Index mit Hilfe von MakeIndex
ordentlich formatiert wird.
Die Datei ist beim Aufruf von MakeIndex anzugeben:

\verb#makeindex -lcgs basis.ist #\meta{sourcedatei}

\minisec{Quelltext}

Der \verb#source#-Zweig enthält den Quelltext des Basis-Pakets.
Alle Änderungen sind hier vorzunehmen und die anderen Dateien zu generieren.

\minisec{Stildateien}

Die Stildateien sind die Dateien, die beim \LaTeX-Lauf zur Formatierung genutzt werden.



\section{Die Benutzerschnittstelle des Basis-Stils}

Das \paket{basis}-Paket basiert auf den KOMA-Script-Klassen.
Daher sind diese als Dokumentklasse für Dokumente zu laden,
bevor das \paket{basis}-Paket eingebunden wird:

\verb#\documentclass{scrartcl}# bzw.\ \verb#\documentclass{scrlttr2}#

\subsection{Benutzung}

Um einen Text im Basis-Layout zu setzen, ist es notwendig,
das \paket{basis}-Paket wie folgt zu benutzen:

\verb#\usepackage{basis}#

Dem Paket können Optionen übergeben werden, die in \autoref{sec:Optionen} erläutert werden.

\subsection{Benötigte Pakete}
\label{sec:BenoetigtePakete}

Das \paket{basis}-Paket bindet die Pakete bereits ein, die entweder für das Paket notwendig sind oder für das Schreiben von Papieren hilfreich sind.
Die Pakete werden im folgenden kurz vorgestellt und müssen für die Nutzung des \paket{basis}-Pakets zur Verfügung stehen.

Das heißt, die Pakete müssen vom Anwender auf dem Rechner installiert werden, sonst gibt es Fehlermeldungen.

Eine genauere Beschreibung der einzelnen Pakete ist in der
Dokumentation der Pakete selbst zu finden.
\begin{description}
 \item [array] Nützliche Zusatzdefinitionen für Tabellen.
 \item [babel] Das Sprachpaket von \LaTeX.
 \item [booktabs] Schöne Tabellenlinien.
 \item [fixme] fixme-Befehle, durch Option \option{fixme} einzubinden.
 \item [graphicx] Das Paket ist dafür zuständig, Grafiken auszugeben. Diese können bei der Ausgabe skaliert werden.
 \item [helvet] Helvetica-Schrift
 \item [hyperref] Inhaltsverzeichnis und navigierbare Links.
				(Kann durch die Option \option{nohyper} ausgeschaltet werden.)
 \item [xifthen] Das Paket stellt vereinfachte boolesche Abfragen zur Verfügung.
 \item [inputenc] Das Paket definiert die direkte Eingabe von Sonderzeichen im laufenden Text.
 \item [jurabib] Dieses Paket dient zur Gestaltung von geisteswissenschaftlichen Literaturzitaten und -verzeichnissen.
				(Kann durch die Option \option{nojura} ausgeschaltet werden.)
 \item [longtable] Große Tabellen.
 \item [luximono] Luxi-Mono-Schrift
 \item [makeidx] Dieses Paket dient zur Indexierung von Dokumenten und wird nur geladen, wenn die Option \option{index} gewählt wurde.
 \item [marvosym] Das Paket enthält viele Symbole, die in den normalen Schriftarten fehlen.
				So wird z.\,B.\ das Euro-Zeichen (\EUR) zur Verfügung gestellt.
				Eine Auswahl anderer nützlicher Symbole sind Telefon (\Telefon), Handy (\Mobilefone), Fax (\fax) oder auch email (\Email).
 \item [mathptmx] Times-Schrift
 \item [microtype] ausgeglichenerer Schrriftsatz incl\ Randausgleich
 \item [scrartcl/scrbook/scrreprt/scrlttr2] Die KOMA-Script-Klassen müssen zur Verfü"-gung stehen und genutzt werden, da die Definitionen im \paket{basis}-Stil darauf zurückgreifen.
 \item [scrpage2] Die KOMA-Script-Klasse für selbst definierte Kopf- bzw.\ Fußzeilen.
 \item [setspace] Einstellung eines anderen Zeilenabstands (nur bei entsprechender Option)
\end{description}

\subsection{Optionen}
\label{sec:user:options}

Die Optionen des \paket{basis}-Pakets werden zunächst einzeln erläutert, sie können
auch beliebig kombiniert werden, die geschieht durch Trennung mit Kommata.

\verb#\usepackage[draft, index]{basis}#

Mögliche Optionen: \option{draft}, \option{final}, \option{font}, \option{hypercolor}, \option{hyperdriver}, \option{layout}, \option{nobackaddress}, \option{nofoldmarks}, \option{notitlepage}, \option{oneside}, \option{protokoll}, \option{pagestyle}

\minisec{bewerbung}

Die \option{bewerbung}-Option stellt den Bewerbungsstil ein.

\verb#\usepackage[bewerbung]{basis}#

\minisec{draft}

Die \option{draft}-Option bewirkt, dass das Dokument als Entwurfsdokument gekennzeichnet wird.
Das bedeutet einen fetten Schriftzug "`Entwurf"' und einen Zeitstempel in der Fußzeile.

\verb#\usepackage[draft]{basis}#

\minisec{font}

Die \option{font}-Option sorgt für die Einstellung eines bestimmten Fontschemas.
Mögliche Schemas: \option{charter}, \option{hfold}, \option{mathpazo}, \option{original}, \option{times} (default)

\verb#\usepackage[font=\meta{Schema}]{basis}#\\
\verb#\usepackage[font=mathpazo]{basis}#

\minisec{hypercolor}

Die Option \option{hypercolor} färbt Links in der gewünschten Farbe statt defaultmäßig blau.

\verb#\usepackage[hypercolor=black]{basis}#

\minisec{layout}

Die \option{layout}-Option sorgt für die Einstellung eines bestimmten Brieflayouts.
Mögliche Schemas: \option{bewerbung}, \option{kopfzeile}, \option{infospalte} (default)

\verb#\usepackage[layout=\meta{Schema}]{basbrief}#\\
\verb#\usepackage[layout=kopfzeile]{basbrief}#

\begin{description}
		\item [bewerbung] wie \option{infospalte} ohne Falzmarken und Rücksendeadresse
		\item [kopfzeile] Adressangaben in Kopfzeile
		\item [infospalte] Adressangaben in separater Spalte
\end{description}

\subsection{Neue bzw.\ geänderte Befehle und Umgebungen}
\label{sec:New}

Dieser Abschnitt führt alle Befehle und Umgebungen, die neu hinzugekommen sind oder die sich in der Bedienung geändert haben, auf und erläutert sie.

\subsubsection{Allgemeine Befehle bzw.\ Änderungen}
\label{sec:New:Allgemein}

\DescribeMacro{\EUR}
Der Befehl \verb#\EUR# gibt das Euro-Symbol (\EUR) aus.
Der Befehl stammt aus dem \paket{marvosym}-Paket.
\begin{einspiel}{Einsatz}
 \>\verb#\EUR#\\
 \>\verb#\EUR{}#
\end{einspiel}
\begin{einspiel}{Beispiel}
 \>\verb#Das neue Währungssymbol ist \EUR.#\\
 \>\verb#Das Symbol \EUR{} ist nicht schön.#
\end{einspiel}

\DescribeMacro{\EURdig}
Der Befehl \verb#\EURdig# gibt das Euro-Symbol in der Breite der Zahlen des
Fonts (15~\EURdig) aus.
Der Befehl stammt aus dem \paket{marvosym}-Paket und
soll die Formatierung von Symbol und Zahlen in Tabellen erleichtern.
\begin{einspiel}{Einsatz}
 \>\verb#\EURdig#\\
 \>\verb#\EURdig{}#
\end{einspiel}
\begin{einspiel}{Beispiel}
 \>\verb#Du hast 15 \EURdig.#\\
 \>\verb#Ich bekomme 15 \EURdig{} von Dir.#
\end{einspiel}

\DescribeMacro{\meta}
Der Befehl \verb#\meta# setzt den übergebenen Text als \meta{Metatext}.
Das
bedeutet, dass spitze Klammern um den schräg gestellten Text geschrieben werden.
\begin{einspiel}{Einsatz}
 \>\verb#\meta{#\meta{Text}\verb#}#
\end{einspiel}
\begin{einspiel}{Beispiel}
 \>\verb#\meta{Metatext}#
\end{einspiel}

\DescribeMacro{\textsubscript}
Der Befehl \verb#\textsubscript# setzt den übergebenen Text \textsubscript{tiefergestellt}.
Er ist das Pendant zu dem von \LaTeX\ bereitgestellten \verb#\textsuperscript#-Befehl.
\begin{einspiel}{Einsatz}
 \>\verb#\textsubscript{#\meta{Text}\verb#}#
\end{einspiel}
\begin{einspiel}{Beispiel}
 \>\verb#CO\textsubscript{2}#
\end{einspiel}

\subsubsection{Zeitangaben}
\label{sec:New:Zeitangaben}

\DescribeMacro{\datum}
Dieser Befehl gibt das Datum in der Form \emph{tt.\,mm.\ jjjj} aus.
\begin{einspiel}{Einsatz}
 \>\verb#\datum#
\end{einspiel}
\begin{einspiel}{Beispiel}
 \>\verb#Heute ist der \datum.#
\end{einspiel}

\DescribeMacro{\zeit}
Dieser Befehl gibt die Zeit in der Form \emph{hh:mm} aus.
\begin{einspiel}{Einsatz}
 \>\verb#\zeit#
\end{einspiel}
\begin{einspiel}{Beispiel}
 \>\verb#Es ist \zeit\ Uhr.#
\end{einspiel}

\DescribeMacro{\zeitstempel}
Dieser Befehl sorgt dafür, dass in der Fußzeile ein Zeitstempel eingebracht
wird.
Der optionale Parameter dient zur Eingabe eigener Texte, die
in die Fußzeile gebracht werden sollen.
\begin{einspiel}{Einsatz}
 \>\verb#\zeitstempel#\\
 \>\verb#\zeitstempel[#\meta{text}\verb#]#
\end{einspiel}
\begin{einspiel}{Beispiel}
 \>\verb#\zeitstempel#\\
 \>\verb#\zeitstempel[Uhrzeit: \zeit]#\\
\end{einspiel}

\DescribeMacro{\zeitspanne}
Dieser Befehl gibt die übergebenen Parameter als Zeitspanne aus.
Der optionale Parameter dient zur Eingabe des Beginns der Zeitspanne, der obligatorische Parameter enthält das Ende der Zeitspanne.
\begin{einspiel}{Einsatz}
 \>\verb#\zeitspanne[#\meta{start}\verb#]{#\meta{ende}\verb#}#
\end{einspiel}
\begin{einspiel}{Beispiel}
 \>\verb#\zeitspanne{seit 2009}#\\
 \>\verb#\zeitspanne[2008]{2009}#\\
\end{einspiel}

\subsubsection{Die Titelseite}
\label{sec:New:Titelseite}

\DescribeMacro{\title}
Der Befehl \verb#\title# gibt den angegebenen Text als Titel des Haupttitels
auf dem Titelblatt aus.
Die Angabe ist optional und wird bei \verb#\maketitle#
benutzt.
Zusätzlich zu dem normalen \verb#\title#-Befehl von \LaTeX\ kann ein
optionaler Parameter angegeben werden, der einen Kurztext enthält, der
in die Fußzeile eingetragen wird.
\begin{einspiel}{Einsatz}
 \>\verb#\title{#\meta{text}\verb#}#\\
 \>\verb#\title[#\meta{kurztext}\verb#]{#\meta{text}\verb#}#
\end{einspiel}
\begin{einspiel}{Beispiel}
 \>\verb#\title{Der basis-Stil}#\\
 \>\verb#\title[Mein Stil]{Der basis-Stil}#
\end{einspiel}

\DescribeMacro{\subtitle}
Der Befehl \verb#\subtitle# gibt den angegebenen Text als Untertitel des
Haupttitels auf dem Titelblatt aus.
Die Angabe ist optional und
wird bei \verb#\maketitle# benutzt.
\begin{einspiel}{Einsatz}
 \>\verb#\subtitle{#\meta{text}\verb#}#
\end{einspiel}
\begin{einspiel}{Beispiel}
 \>\verb#\subtitle{Ein \LaTeX{}-Stil angepasst}#
\end{einspiel}

\DescribeMacro{\strasse}
Der Befehl \verb#\strasse# gibt den angegebenen Text als Straße auf dem
Titelblatt aus.
Die Angabe ist optional und wird bei \verb#\maketitle#
benutzt.
Die Ausgabe auf der Titelseite erfolgt nur bei gewählter
\verb#titlepage#-Option, d.\,h.\ bei einer extra Titelseite.
\begin{einspiel}{Einsatz}
 \>\verb#\strasse{#\meta{text}\verb#}#
\end{einspiel}
\begin{einspiel}{Beispiel}
 \>\verb#\strasse{Richard-Sorge-Straße~76}#
\end{einspiel}

\DescribeMacro{\plz}
Der Befehl \verb#\plz# ist analog zum Befehl \verb#\strasse#.

\DescribeMacro{\ort}
Der Befehl \verb#\ort# ist analog zum Befehl \verb#\strasse#.

\DescribeMacro{\telefon}
Der Befehl \verb#\telefon# ist analog zum Befehl \verb#\strasse#.

\DescribeMacro{\email}
Der Befehl \verb#\email# gibt den angegebenen Text als email-Adresse
auf dem Titelblatt aus.
Die Angabe ist optional und wird bei \verb#\maketitle# benutzt.
\begin{einspiel}{Einsatz}
 \>\verb#\email{#\meta{mailadresse}\verb#}#
\end{einspiel}
\begin{einspiel}{Beispiel}
 \>\verb#\email{ekkart@ekkart.de}#
\end{einspiel}

\DescribeMacro{\adresszusatz}
Der Befehl \verb#\adresszusatz# ist analog zum Befehl \verb#\strasse#.

\DescribeMacro{\titelzusatz}
Der Befehl \verb#\titelzusatz# setzt den übergebenen Text in
die rechte untere Ecke der Titelseite.

\subsubsection{Literaturverzeichnis}
\label{sec:New:Literatur}

Das Literaturverzeichnis wurde derart umgestaltet, dass die Überschrift numeriert ist.
Weiterhin wird die Überschrift in die Kopfzeile eingetragen.
Außerdem wird ein Label \verb#sec:Literatur# angelegt,
das auf das Literaturverzeichnis verweist.

\DescribeMacro{\literatur}
Zur Vereinfachung wurde das Makro
\verb#literatur# angelegt, das den Aufruf der entsprechenden \LaTeX-Befehle kapselt.
Der zu übergebende Parameter bezeichnet die Datei, die die Literaturangaben
enthält, ohne Dateiendung.
Als Stildatei wird \verb#jurabib.bst# oder der optionale Parameter angenommen.
\begin{einspiel}{Einsatz}
 \>\verb#\literatur{#\meta{dateiname}\verb#}#
\end{einspiel}
\begin{einspiel}{Beispiel}
 \>\verb#\literatur{kleinod}#
\end{einspiel}

\subsubsection{Index}
\label{sec:New:Index}

Der Index wurde analog zum Literaturverzeichnis derart umgestaltet,
dass die Überschrift numeriert ist.
Weiterhin wird die Überschrift in die Kopfzeile eingetragen.
Außerdem wird ein Label \verb#sec:Index# angelegt, das auf den Index verweist.

Wenn die Option \option{index} angegeben wurde, wird das Paket \paket{makeidx}
geladen, der Befehl \verb#makeindex# bereitet die Indexierung vor und die
Befehle \verb#nindex# sowie \verb#eindex# werden definiert, um die Anwendung
des Index einfach zu gestalten.
Der Index wird wie gewohnt mit \verb#printindex# ausgegeben.

Die Befehle \verb#nindex# und \verb#eindex# sollen die Erstellung
eines Index vereinfachen.
Die originale Indexierung mit Hilfe des \verb#index#-Befehls kann
weiterhin verwendet werden.

\DescribeMacro{\nindex}
Der Befehl \verb#nindex#, \emph{normal index}, trägt den angegebenen Parameter als
Schlagwort in den Index ein und gibt das Wort anstelle des Befehls im Text aus.

Der optionale Parameter dient dazu, den nichtoptionalen Parameter als Unterpunkt
des optionalen Parameters zu kennzeichnen.
\begin{einspiel}{Einsatz}
 \>\verb#\nindex{#\meta{begriff}\verb#}#\\
 \>\verb#\nindex[#\meta{oberbegriff}\verb#]{#\meta{begriff}\verb#}#
\end{einspiel}
\begin{einspiel}{Beispiel}
 \>\verb#\nindex{LaTeX}#\\
 \>\verb#\nindex[Textverarbeitung]{LaTeX}#
\end{einspiel}

\DescribeMacro{\eindex}
Der Befehl \verb#eindex#, \emph{emphasized index},  trägt den angegebenen Parameter
als Schlagwort in den Index ein und gibt das Wort anstelle des Befehls im Text aus.
Außerdem hebt er die Seitenzahl im Index mittels \verb#emph# hervor.

Der optionale Parameter dient dazu, den nichtoptionalen Parameter als
Unterpunkt des optionalen Parameters zu kennzeichnen.
\begin{einspiel}{Einsatz}
 \>\verb#\eindex{#\meta{begriff}\verb#}#\\
 \>\verb#\eindex[#\meta{oberbegriff}\verb#]{#\meta{begriff}\verb#}#
\end{einspiel}
\begin{einspiel}{Beispiel}
 \>\verb#\eindex{LaTeX}#\\
 \>\verb#\eindex[Textverarbeitung]{LaTeX}#
\end{einspiel}

\subsubsection{Vortragsdokumentation}
\label{sec:New:Vortragsdokumentation}

\DescribeMacro{\insertslide}
Der Befehl \verb#insertslide# fügt das Bild einer Folie ein.
Genau gesagt, wird ein Bild rechtsseitig gerahmt mit einer anzugebenden Skalierung eingebunden.
Die Einbindung erfolgt über den \verb#includegraphics#-Befehl, die Skalierungsangabe ist dementsprechend zu wählen.
Die Skalierung ist der erste Parameter, der Präfix des Bildnamens der zweite.

\begin{einspiel}{Einsatz}
 \>\verb#\insertslide{#\meta{skalierung}\verb#}{#\meta{präfix}\verb#}#
\end{einspiel}
\begin{einspiel}{Beispiel}
 \>\verb#\insertslide{width=.3\textwidth}{slide}#\\
 \>\verb#\insertslide{angle=45,width=.2\textwidth}{img}#\\
\end{einspiel}

\DescribeMacro{\nextslide}
Der Befehl \verb#nextslide# kapselt den Aufruf von \verb#insertslide# mit für OpenOffice"=Folien günstigen Werten.
Die Skalierung wird auf 30\,der Textbreite gesetzt, die Dateien müssen mit \emph{slide} beginnen.
Außerdem wird der Folienzähler um eins erhöht.

\begin{einspiel}{Einsatz}
 \>\verb#\nextslide#
\end{einspiel}

\DescribeMacro{\nextslidesilent}
Der Befehl \verb#nextslidesilent# erhöht den Folienzähler um eins, ohne die entsprechende Folie auszugeben.
Damit können z.\,B.\ für die Dokumentation unwichtige Folien übersprungen werden.

\begin{einspiel}{Einsatz}
 \>\verb#\nextslidesilent#
\end{einspiel}

%^^A Ende Nutzerdokumentation

