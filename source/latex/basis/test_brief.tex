\documentclass[ngerman]{scrlttr2}

\usepackage{basbrief}
%\usepackage[draft|final, font=charter|hfold|mathpazo|original|times, hypercolor=<color>, hyperdriver=<driver>, layout=bewerbung|kopfzeile|infospalte, oneside, pagestyle=beides|fuss|leer]{basbrief}

\author{Autorname}

\strasse{Straße}
\plz{PLZ}
\ort{Ort}
\telefon{(030) 1\,22\,33\,44}
\email{abc@abc.de}
\adresszusatz{Adresszusatz}

\usepackage{lipsum}

\begin{document}

	% hier den Empfänger (\\-getrennt) eintragen
	\begin{letter}{Empfänger\\Empfängerstraße\\Empfängerort}

		%\setkomavar{subject}{Betreff} % Betreff
		%\setkomavar{date}{27.\ November 2013} % Datum (default: aktuelles Datum)
		%\setkomavar{signature}{} % Unterschrift (default: Name)

		%\setkomavar{yourref}{} % Geschäftszeile: Ihr Zeichen
		%\setkomavar{yourmail}{} % Geschäftszeile: Ihr Schreiben vom
		%\setkomavar{myref}{} % Geschäftszeile: Unser Zeichen
		%\setkomavar{invoice}{} % Geschäftszeile: Rechnungsnummer

		%\newkomavar*[Meins]{myadd} % Geschäftszeile: neues Feld myadd mit Titel "Meins"
		%\setkomavar{myadd}{Inhalt} % Geschäftszeile: neues Feld

		\opening{Hallo \dots,}

			\lipsum

		\closing{Mit freundlichen Grüßen,}

		\ps PS: Post-Scriptum
		% \setkomavar*{enclseparator}{Anlage} % für genau eine Anlage
		\encl{Anlage 1, Anlage 2\\Anlage 3}
		\cc{Empfänger 1, Empfänger 2\\Empfänger 3}

	\end{letter}

\end{document}

