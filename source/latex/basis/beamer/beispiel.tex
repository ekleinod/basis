\mode<presentation> {
	%\usetheme{Singapore}
	%\usetheme{Berkeley}
	%\usetheme{Rochester}
	\usetheme{basis}

	%\usecolortheme{dove}
	\usecolortheme{seahorse}

	%\usepackage{beamerthemesplit}
	
	\usepackage{helvet}

	\setbeamercovered{transparent}
}

\mode<handout> {
	\usepackage{pgfpages}
	%\pgfpagesuselayout{2 on 1}[a4paper,border shrink=5mm]
	\pgfpagesuselayout{4 on 1}[a4paper,landscape,border shrink=5mm]
}

\usepackage[ngerman]{babel}
\usepackage[utf8]{inputenc}
\usepackage[T1]{fontenc}

\title[Dissertation]{Verteilte Modellierung und virtuelle Integration von überlappenden Komponenten}

\subtitle{Ein aspektorientierter Ansatz am Beispiel von Funktionsarchitekturen für eingebettete Systeme im Automobil.}

\author{Ekkart Kleinod}
\institute[FoKo SWT]{Forschungskolloquium SWT TU-Berlin}
\date{19. Mai 2011}


% Thema der Dokumenteigenschaften
\subject{AO, eingebettete Systeme, Architekturmodellierung}

% Folgendes sollte gelöscht werden, wenn man nicht am Anfang jedes
% Unterabschnitts die Gliederung nochmal sehen möchte.
%\AtBeginSubsection[]
%{
%	\begin{frame}<beamer>{Agenda}
%		\tableofcontents[currentsection,currentsubsection]
%	\end{frame}
%}

\AtBeginSection[]
{
	%\addtocounter{framenumber}{-1}
	\begin{frame}<beamer>{Agenda}
		\tableofcontents[currentsection, hideallsubsections]
	\end{frame}
}


\begin{document}

	\mode<article> {
		\maketitle
	}

	\begin{frame}
		\titlepage
%		\setcounter{framenumber}{-1}
	\end{frame}

	\section*{Agenda}

	\begin{frame}{Agenda}
		\tableofcontents[hideallsubsections]
		% Die Option [pausesections] könnte nützlich sein.
	\end{frame}


	\section{Ziel}

	\subsection{Beschreibung}

	\begin{frame}{\insertsectionhead}
		Diese Dissertation beschäftigt sich mit verteilter Modellierung und virtueller Integration von überlappenden Komponenten im Bereich des Automobilbaus.
		Ziel ist, eine Methode zu schaffen, mit der überlappende Komponenten modelliert werden können.
		Dazu werden Beschreibungsmittel und Methoden definiert.
		\note[item]{vM definieren}
		\note[item]{üK definieren}
	\end{frame}
	
	\note[itemize]{
		\item eine Anmerkung
		\item eine zweite Anmerkung
	}

	\note[enumerate]{
		\item eine Anmerkung
		\item eine zweite Anmerkung
	}

	\subsection{Prinzipdarstellung}

	\begin{frame}{\insertsectionhead}{\insertsubsectionhead}
		\hfil\includegraphics[width=.9\textwidth]{../dissertation/dissertation/figures/principle02}\hfil
	\end{frame}

	\section{Ausgangslage}

	\subsection{VEIA-Referenzprozess der Systemmodellierung, Fokus logische Architektur}

	\begin{frame}{\insertsectionhead}{\insertsubsectionhead}
		\hfil\includegraphics[width=.7\textwidth]{../dissertation/dissertation/figures/figure01}\hfil
	\end{frame}

	\section{Problem}

	\subsection{Nichtfunktionale Änderungen, querliegende Funktionen}

	\begin{frame}{\insertsectionhead}{\insertsubsectionhead}
		\hfil\includegraphics[width=.7\textwidth]{../dissertation/dissertation/figures/figure08}\hfil
	\end{frame}

	\begin{itemize}
		\item Nebeneffekte bei Modellierung
		\item Hardware beeinflusst Logik (z.\,B.\ Signalaufbereitung), muss modelliert werden
		\item Diagnose in allen Modulen
	\end{itemize}

	\subsection{Modellierung = Änderung}

	\begin{frame}{\insertsectionhead}{\insertsubsectionhead}
		\hfil\includegraphics[width=.7\textwidth]{../dissertation/dissertation/figures/figure02}\hfil
	\end{frame}

	\begin{itemize}
		\item eigentliches Problem -- Änderungen werden nicht aufgehoben
	\end{itemize}

	\subsection{Änderungen bewahren}

	\begin{frame}{\insertsectionhead}{\insertsubsectionhead}
		\hfil\includegraphics[width=.7\textwidth]{../dissertation/dissertation/figures/figure04}\hfil
	\end{frame}

	\begin{itemize}
		\item Änderungen, Entwicklung
		\item Funktionalität bleibt als Artefakt erhalten
		\item Virtuelle Integration
		\item Echte verteilte Modellierung
		\item Zusammenführen automatisch
		\item Implementierung, Test, etc.\ wie bisher
	\end{itemize}

	Mögliche Lösungen:

	\begin{itemize}
		\item Nichtmodellierung
		\item Versionierung
		\item Auszeichnung durch Relationen zu Featuremodellen
		\item Trennung der Originalmodelle und Relationen zwischen ihnen
	\end{itemize}

	\subsection{Modellieren und mischen}

	\begin{frame}{\insertsectionhead}{\insertsubsectionhead}
		\hfil\includegraphics[width=.7\textwidth]{../dissertation/dissertation/figures/figure05}\hfil
	\end{frame}

	\section{Lösung}

	\subsection{Beispiel: Einfaches E/E-System -- Überlappungen}

	\begin{frame}{\insertsectionhead}{\insertsubsectionhead}
		\hfil\includegraphics[width=.7\textwidth]{../dissertation/dissertation/figures/figure09}\hfil
	\end{frame}

	\subsection{Beispiel: Einfaches E/E-System -- Angestrebte Modellierung}

	\begin{frame}{\insertsectionhead}{\insertsubsectionhead}
		\hfil\includegraphics[width=\textwidth]{../dissertation/dissertation/figures/figure10}\hfil
	\end{frame}

	\subsection{Beispiel CBS: Trennung der Modellierung}

	\begin{frame}{\insertsectionhead}{\insertsubsectionhead}
		\hfil\includegraphics[width=\textwidth]{../dissertation/dissertation/figures/casestudy06}\hfil
	\end{frame}

	\begin{itemize}
		\item Funktionalitäten zuordnen
	\end{itemize}

	\subsection{Beispiel CBS: Erste Identität}

	\begin{frame}{\insertsectionhead}{\insertsubsectionhead}
		\hfil\includegraphics[width=\textwidth]{../dissertation/dissertation/figures/casestudy07}\hfil
	\end{frame}

	\subsection{Neue Metamodellartefakte}

	\begin{frame}{\insertsectionhead}{\insertsubsectionhead}
		\begin{itemize}
			\item abstrakte Komponenten
			\item Aspektrelationen bestehend aus Aspektlinks (\textit{identity}, \textit{inner}, \textit{copy}, \textit{replace})
			\item Kardinalität für Komponenten und Ports
		\end{itemize}
	\end{frame}

	\subsection{Beispiel CBS: Neue Modellierungsmöglichkeiten}

	\begin{frame}{\insertsectionhead}{\insertsubsectionhead}
		\hfil\includegraphics[width=\textwidth]{../dissertation/dissertation/figures/casestudy12}\hfil
	\end{frame}

	\subsection{Beispiel CBS: Nach der Mischung}

	\begin{frame}{\insertsectionhead}{\insertsubsectionhead}
		\hfil\includegraphics[width=.8\textwidth]{../dissertation/dissertation/figures/casestudy13}\hfil
	\end{frame}

	\subsection{Anwendungsfälle}

	\begin{frame}{\insertsectionhead}{\insertsubsectionhead}
		\begin{itemize}
			\item Überlappungen
			\item Verfeinerungen
			\item Mustermodellierung
		\end{itemize}
	\end{frame}

	\subsection{Anwendungsfall Verfeinerung}

	\begin{frame}{\insertsectionhead}{\insertsubsectionhead}
		\hfil\includegraphics[width=.4\textwidth]{../dissertation/dissertation/figures/casestudy15}\hfil\\[.1\textheight]
		\hfil\includegraphics[width=.6\textwidth]{../dissertation/dissertation/figures/casestudy16}\hfil
	\end{frame}

	\subsection{Anwendungsfall Mustermodellierung}

	\begin{frame}{\insertsectionhead}{\insertsubsectionhead}
		\hfil\includegraphics[width=.5\textwidth]{../dissertation/dissertation/figures/casestudy22}\hfil\\[.1\textheight]
		\hfil\includegraphics[width=.7\textwidth]{../dissertation/dissertation/figures/casestudy23}\hfil
	\end{frame}

	\section{Stand der Dissertation}

	\subsection{Geplante Bestandteile}

	\begin{frame}{\insertsectionhead}{\insertsubsectionhead}
		\begin{itemize}
			\item Modellierungsmittel (Metamodell) für überlappende Komponenten
			\item Methodisches Vorgehen in der Praxis
			\item Beispiele zur Anwendung
			\item Algorithmus zur Mischung der Modelle
			\item prototypische Implementierung Mischalgorithmus
		\end{itemize}
	\end{frame}

	\begin{itemize}
		\item Modellierungsmittel (Metamodell) für überlappende Komponenten
					\begin{itemize}
						\item Basis: Ideen der Aspektorientierung
						\item Aspekte als Lösungsmuster
						\item Instanzen als Lösungen
						\item Relationen innerhalb der Komponentenmodelle
						\item Semantik der Relationen
					\end{itemize}
		\item Methodisches Vorgehen in der Praxis
					\begin{itemize}
						\item Aufwand vs. Nutzen
						\item Komplexität der Modellierung
						\item Best Practices
					\end{itemize}
		\item Beispiele zur Anwendung
					\begin{itemize}
						\item Zeigen der Praxistauglichkeit
						\item Modellierungsmöglichkeiten
					\end{itemize}
		\item Algorithmus zur Integration der Modelle
					\begin{itemize}
						\item Verweben der Einzelmodelle zu einem Gesamtmodell
						\item Umsetzung der Relationssemantik
						\item Betrachtungen zur Reihenfolge der Verwebung
					\end{itemize}
	\end{itemize}

	\subsection{Prototyp aXBench}

	\begin{frame}{\insertsectionhead}{\insertsubsectionhead}
		\hfil\includegraphics[width=.9\textwidth]{../dissertation/dissertation/figures/axbench}\hfil
	\end{frame}

	\subsection{Aktuell}

	\begin{frame}{\insertsectionhead}{\insertsubsectionhead}
		\begin{itemize}
			\item Lösung beschrieben
			\item Metamodell
			\item Text komplett aufgeschrieben, derzeit in zweiter Überarbeitung
			\item \textit{aXLang} (Sprache) erweitert
			\item Implementierung \textit{identity} fast fertig, Rest noch nicht
			\item Offen für Hinweise auf ähnliche Arbeiten
		\end{itemize}
	\end{frame}
	
	% do not count appendix
	\newcounter{finalframe}
	\setcounter{finalframe}{\value{framenumber}}
	\setbeamertemplate[footline]{default}

	\appendix
	
	\section{VEIA-Referenzprozess der Systemmodellierung}

	\begin{frame}{\insertsectionhead}
		\hfil\includegraphics[width=.7\textwidth]{../dissertation/dissertation/figures/reference}\hfil
	\end{frame}

	\section{Metamodelle}

	\subsection{Ursprüngliches (vereinfachtes) Metamodell}

	\begin{frame}{\insertsectionhead}{\insertsubsectionhead}
		\hfil\includegraphics[width=.7\textwidth]{../dissertation/dissertation/figures/metamodel}\hfil
	\end{frame}

	\subsection{Geändertes Metamodell}

	\begin{frame}{\insertsectionhead}{\insertsubsectionhead}
		\hfil\includegraphics[width=\textwidth]{../dissertation/dissertation/figures/metamodel_new}\hfil
	\end{frame}

	\section{Modellierungsmöglichkeiten}

	\subsection{Instanziierungsmöglichkeiten}

	\begin{frame}{\insertsectionhead}{\insertsubsectionhead}
		\hfil\includegraphics[width=.9\textwidth]{../dissertation/dissertation/figures/cardinality2}\hfil
	\end{frame}

	\subsection{Gemeinsame Kardinalität}

	\begin{frame}{\insertsectionhead}{\insertsubsectionhead}
		\includegraphics[width=.35\textwidth]{../dissertation/dissertation/figures/cardinality3a}
		\hfill
		\includegraphics[width=.35\textwidth]{../dissertation/dissertation/figures/cardinality3b}
	\end{frame}

	\subsection{Reihenfolge der Instanziierung}

	\begin{frame}{\insertsectionhead}{\insertsubsectionhead}
		\hfil\includegraphics[width=.8\textwidth]{../dissertation/dissertation/figures/cardinality4}\hfil
	\end{frame}

	\section{Beispiel CBS}

	\subsection{Organisationseinheiten}

	\begin{frame}{\insertsectionhead}{\insertsubsectionhead}
		\hfil\includegraphics[width=.7\textwidth]{../dissertation/dissertation/figures/casestudy05}\hfil
	\end{frame}

	\begin{itemize}
		\item keine geeignete Dekomposition, unabhängig von Kriterium
		\item damit keine getrennte Modellierung möglich -- Abstimmungsprobleme
	\end{itemize}

	\subsection{CBS, Motormanagement, Anzeigen}

	\begin{frame}{\insertsectionhead}{\insertsubsectionhead}
		\hfil\includegraphics[width=.7\textwidth]{../dissertation/dissertation/figures/casestudy03}\hfil
	\end{frame}

	\subsection{Anwendungsfall Vollständige Mustermodellierung}

	\begin{frame}{\insertsectionhead}{\insertsubsectionhead}
		\hfil\includegraphics[width=.5\textwidth]{../dissertation/dissertation/figures/casestudy22a}\hfil\\[.1\textheight]
		\hfil\includegraphics[width=.7\textwidth]{../dissertation/dissertation/figures/casestudy23}\hfil
	\end{frame}

	\subsection{Anwendungsfall Redundante Verbindung}

	\begin{frame}{\insertsectionhead}{\insertsubsectionhead}
		\hfil\includegraphics[width=.5\textwidth]{../dissertation/dissertation/figures/casestudy20a}\hfil\\[.1\textheight]
		\hfil\includegraphics[width=.7\textwidth]{../dissertation/dissertation/figures/casestudy21}\hfil
	\end{frame}

	\subsection{CBS-Musterarchitektur}

	\begin{frame}{\insertsectionhead}{\insertsubsectionhead}
		\hfil\includegraphics[width=\textwidth]{../dissertation/dissertation/figures/casestudy24}\hfil
	\end{frame}

	\subsection{Hauptuntersuchung und Motoröl}

	\begin{frame}{\insertsectionhead}{\insertsubsectionhead}
		\hfil\includegraphics[width=.8\textwidth]{../dissertation/dissertation/figures/casestudy25}\hfil
	\end{frame}

	\subsection{Hauptuntersuchung und Motoröl (Systemarchitektur)}

	\begin{frame}{\insertsectionhead}{\insertsubsectionhead}
		\hfil\includegraphics[width=.8\textwidth]{../dissertation/dissertation/figures/casestudy27}\hfil
	\end{frame}

	%% do not count appendix
	\setcounter{framenumber}{\value{finalframe}}

\end{document}

